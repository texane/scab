\documentclass[a4paper, 11pt]{article}

\usepackage{graphicx}
\usepackage{graphics}
\usepackage{verbatim}
\usepackage{listings}
\usepackage{color}
\usepackage{bytefield}

\begin{document}

\title{SCAB: serial to CAN bridge}
\author{IGREBOT team}
\date{}

\maketitle


\newpage
\tableofcontents
\addtocontents{toc}{\protect\setcounter{tocdepth}{1}}


\newpage
\begin{abstract}
This document describes the SCAB api library and the DSPIC33F firmware.
\end{abstract}


\newpage
\section{Introduction}

\subsection{Overview}
\begin{figure}[]
\centering
\includegraphics[scale=0.5]{./dia/scab_network/main.jpeg}
\caption{serial to CAN bridging}
\label{scab_bridging}
\end{figure}

\paragraph{}
Usually, PCs do not have the hardware interfaces required to communicate on a
CAN network. SCAB is an opensource project to add CAN connectivity to PCs by
bridging a serial port. To do so, SCAB provides the following:
\begin{itemize}
\item a firmware to be flashed on a DSPIC33F board,
\item a programming interface implemented in a library used by host applications.
\end{itemize}

\subsection{Availability}
\paragraph{}
The project is maintained in a GIT repository:
\begin{center}
https://github.com/texane/scab
\end{center}
\paragraph{}
In the remaining of this document, the expression:
\begin{center}
\$SCAB\_REPO\_DIR
\end{center}
denotes the directory where the repository was cloned.

\subsection{Dependencies}
\paragraph{}
The project depends on the following softwares:
\begin{itemize}
\item a working LINUX system with standard GNU tools,
\item MPLABX version 1.0 .
\end{itemize}
Note that WINDOWS and MACOSX are not yet supported.


\newpage
\section{Host application programming interface}

\subsection{Overview}
\paragraph{}
The source code is located in:
\begin{center}
\$SCAB\_REPO\_DIR/src/api
\end{center}

\paragraph{}
The API is shipped as a standalone static library and can be built using:\\
\begin{small}
\lstset{commentstyle=\color{blue}}
\lstset{language=C}
\begin{lstlisting}[frame=tb]
cd $SCAB_REPO_DIR/build/api ;
make ;
\end{lstlisting}
\end{small}

\paragraph{}
It produces the file:
\begin{center}
\$SCAB\_REPO\_DIR/build/api/libscab\_api.a
\end{center}

If a GNU toolchain is used, the library is linked in a client application
by adding the following flags to command line:\\
\begin{small}
\lstset{commentstyle=\color{blue}}
\lstset{language=C}
\begin{lstlisting}[frame=tb]
-L$SCAB_REPO_DIR/build/api -lscab_api
\end{lstlisting}
\end{small}

\subsection{Interface documentation}
\paragraph{}
\begin{small}
\lstset{commentstyle=\color{blue}}
\lstset{language=C}
\begin{lstlisting}[frame=tb]
int scab_open(scab_handle_t**, const char*);
int scab_close(scab_handle_t*);
int scab_sync_serial(scab_handle_t*);
int scab_read_frame(scab_handle_t*, uint16_t*, uint8_t*);
int scab_write_frame(scab_handle_t*, uint16_t, const uint8_t*);
int scab_enable_bridge(scab_handle_t*);
int scab_disable_bridge(scab_handle_t*);
int scab_set_can_filter(scab_handle_t*, uint16_t, uint16_t);
int scab_clear_can_filter(scab_handle_t*);
int scab_get_handle_fd(scab_handle_t*);
\end{lstlisting}
\end{small}

\subsection{Example program}
\paragraph{}
TODO

\subsection{Limitations}
\paragraph{}
TODO

\newpage
\section{Device firmware}

\subsection{Overview}
\paragraph{}
The firmware is a piece of software put on a DSPIC33F board to perform the forwarding
of frame to and from the host PC an the CAN network.

\paragraph{}
The source code is located in:
\begin{center}
\$SCAB\_REPO\_DIR/src/device
\end{center}

\paragraph{}
Assuming that MPLABX version 1.0 has been installed with the default paths, the firmware
can be compiled using:\\
\begin{small}
\lstset{commentstyle=\color{blue}}
\lstset{language=C}
\begin{lstlisting}[frame=tb]
cd $SCAB_REPO_DIR/build/device.X ;
make ;
\end{lstlisting}
\end{small}

\paragraph{}
It produces the file:
\begin{center}
\$SCAB\_REPO\_DIR/build/device.X/dist/default/production/device.X.production.hex
\end{center}
which is used to program the DSPIC33F flash.

\subsection{Limitations}
\paragraph{}
TODO

\newpage
\section{Protocol between host and device}

\subsection{Overview}
\paragraph{}
All the symbolic constants used in this section can be found in the file:
\begin{center}
\$SCAB\_REPO\_DIR/src/common/scab\_common.h
\end{center}

\paragraph{}
All the packets share the same basic format and are of the same fixed length
SCAB\_CMD\_SIZE. The packets can be sorted in 2 groups:
\begin{itemize}
\item bridge management commands: set parameters such as link speeds, CAN filters ...
\item frame forwarding: actual data frame forwarding.
\end{itemize}

\subsection{Packet formats}
\paragraph{}
TODO

\end{document}
